\documentclass[conference]{IEEEtran}

% Packages
\usepackage{cite}
\usepackage{amsmath,amssymb,amsfonts}
\usepackage{algorithmic}
\usepackage{graphicx}
\usepackage{textcomp}
\usepackage{xcolor}
\usepackage{todonotes}

% Title
\title{Internet of Things in the Automobile Industry}

% Authors
\author{\IEEEauthorblockN{Henry Kniß}
\IEEEauthorblockA{Fachbereich 02, University of Applied Sciences\\
Frankfurt, Germany\\
Email: {kniss@stud.fra-uas.de}}}

\begin{document}

\maketitle

\begin{abstract}




The Internet of Things (IoT) has had a significant impact on the automobile sector, influencing production, operations, and daily product usage. One notable advancement enabled by IoT is automated driving through car2X communication. Car-to-X communication enables vehicles to interact with other vehicles (C2C) as well as with infrastructure (C2I), improving safety and efficiency on the roads. Another emerging technology in the automotive industry is digital twins, which integrate AI into vehicles and the manufacturing process. Digital twins create virtual replicas of physical assets, allowing for real-time monitoring, analysis, and optimization. They have the potential to transform, or are already transforming the industry by enabling holistic systems and optimizing traffic flow in cities. While advancements in RFID technology and predictive maintenance have contributed to cost savings and increased efficiency in the manufacturing process. This paper aims to provide an overview of these emerging technologies and discusses the challenges they face in manufacturing and automated driving concepts. Its goal is to serve as a reference for further research in this field, offering insights into the potential and implications of IoT advancements in the automobile industry.
\end{abstract}

\textbf{Keywords:} IoT, Car2X, Automobile , Survey

\section{Introduction}
Over the past 60 years since the first introduction of the Internet, several approaches were made to enhance connectivity between the internet and the real world. The ever-morphing terminology describing those attempts has changed over the years to finally arrive at "Internet of Things". Nowadays it encompasses a wide array of topics that touch on various other cornerstones of research in the field of computer sciences such as Automation, Artificial Intelligence (AI), and Networks, just to name a few. In the context of Industry 4.0, the topics collected under the umbrella term of IoT are expanded upon and are further developed depending on the specific industrial branch they are used in. In this paper, we will focus on the usage of IoT in the automobile industry. Here several developments in the fields of sensor technology and communication (such as 5G) that have been in the works for years have culminated, giving rise to a wide spectrum of new technologies. "Some of these advancements, such as automated driving resulting from enhanced Car2X communication, are highly visible to the public. Others, like the development and application of digital twins, have remained largely under the public's radar. The Digital Twins itself being a prime example of the interweaving of several topics under the terminology by interconnecting real-time data collection, transmission, analytics, and application through sensors, Network as well as AI. In the context of less publicly present achievements, we will also talk about how the IoT developments have influenced the manufacturing process itself.

To this end, as mentioned above, this paper will first focus on the communication of vehicles with other endpoints (C2X). This contains communication between vehicles (C2C) as well as the communication between a vehicle and the infrastructure (C2I). A lot of the use cases of this technology have been defined in the "Car-2-car communication consortium-manifesto" by Prof. Dr. Matthias Deegener et al \cite{baldessari2007car}. Even though the manifesto itself is from the year 2007, it still provides a valuable entry point into the understanding of the relevance and capabilities of the Car2X principle itself. Afterwards, we will discuss the potential and the current state of the research concerning Digital Twins and their uses in the construction as well in the use of vehicles using the paper "Digital Twins in the Automotive Industry: The Road toward Physical-Digital Convergence" by Dimitrios Piromalis \cite{piromalis2022digital} as a guiding source. Finaly, we will continue with a quick overview of the relevant applications in manufacturing before finishing with a discussion as well as a conclusion about the state of the research.

\section{Methodology}
During the writing of the survey a number of papers where reviewed from different sources across the internet, the main focus where those from the Institute of Electrical and Electronics Engineers (IEEE) website but also from google scholar in general. Most of the discussed articles can be found on multiple websites. When selecting  appropriate articles for this research, there where a number of factors to be considered, including, but not limited to

\begin{enumerate}
    \item Release date
    \item Journal Impact Factor (JIF)
    \item Authority of the author
    \item Relevance of the titel
\end{enumerate}

While the relevance of the relaese date itself is self explanatory the other points require a few more edetails. The JIF is a measure reflecting the average number of citations to recent articles posted. It gives us an indicator for the relevance of the journal and that a given article was published in higher scores, thus indicating higher relevance. The authority of the author was based of a quick search of the author of an article to inidcate whether they had published further articles in the relevant field to indicate their knowledgeability further lending another indicator for relevance. Combining all those factors it was possible to asses whether an certain article was of intrest to the author or not.

Regarding the types of articles reviewed,  case studies and literature reviews where prefreneced due to the broad scope of the topics. Such papers often offered a concise overview, presenting a condensed analysis of the subject at hand.

\section{Car2X Communication}
\label{Se:car2Car}
Car2X communication (C2X) describes the communication of cars and other IoT devices. These can vary in types and divide C2X into further subclasses. The communication with city infrastructure such as traffic lights or Road Side Units (RSU) is described as Car 2 Infrastructure (C2I) while the communication to other vehicles is described as (C2C). But the essence of all these interactions can be formulated as the dissemination of information into an ad hoc network made up of several IoT devices. In the forefront of this general car-based IoT application are the aspects of networking, sensors, and security as well as safety combined to enable the foundation of automated driving. This enablement can in general be described as the goal to which the C2X technology is a gateway. We will focus in the following chapter on the technologies that enable this communication and especially on the challenges the networks are still facing in terms of communication, as well as presenting a few example use cases of the technology in general.

An ad hoc network is a decentralized and spontaneously formed network existing within a certain region around the participants. It is differentiated into so called "Single-Hop" and "Multi-Hop" networks that differ in the amount of nodes a message can travel across. The most commonly used form of an ad hoc network for the C2X communication is the "Vehicular Ad Hoc Network" (VANET).\cite{liu2019congestion} This specialized network is based on the widely used "Mobile Ad Hoc Network" (MANET) but further supports communication on the "Physical Layer" as well as the "Data Link Layer" by using the IEEE802.11p Protocol.
One major challenge inherent to ad hoc networks are the communication errors that can arise by too high or too low node density within the network. While the "Broadcast Storm" occurs at too high node density as a result of the network getting flooded with messages, networks with a too low node density struggle with insufficient data penetration. Since both hinder the ability of the cars to receive critical information about possible risks, they present a great threat to the safety of automated driving.
Naturally, there have been several suggested solutions for solving the issue which can roughly be sorted into 6 categories. \cite{rashid2020overview} \cite{rashid2020reliable}

\begin{table}[h]
\renewcommand{\figurename}{Figure}
\centering
\begin{tabular}{||c c||} 
 \hline
 Categorie & Examples \\ [0.5ex] 
 \hline\hline
 Quality of Service & \cite{wahab2013vanet}\cite{akamatsu2014adaptive}\\ 
 \hline
 Delay based data dissemination & \cite{salvo2013timer}\cite{sospeter2018effective} \\
 \hline
 Probability based data dissemination&\cite{wisitpongphan2007broadcast}\cite{tonguz2010dv} \cite{sospeter2018effective}\\
 \hline
Push based data dissemination & \cite{schwartz2011directional}\\
 \hline
Pull based data dissemination &\cite{mondal2015secure}\\
 \hline
Cluster based data dissemination & \cite{zhang2013smartgeocast}\\ [1ex] 
 \hline
\end{tabular}
\caption{Categories for attempts a data dissemination in VANETS based on \cite{rashid2020reliable} \cite{rashid2020overview}}
\label{Routingtabel}
\end{table}

Quality of Service (QoS) describes a set of service requirements the network must meet when transporting a data stream. As such, approaches in this category aim to meet certain predefined requirements. The delay-based approach, on the other hand, relies on the definition of relay nodes to send messages through the network, suppressing distribution by nodes that were not selected as relays. In probability-based approaches, on the other hand, those messages are forwarded with priority when they have the greatest chance of reducing network load. The responsibility to keep the network load low is transferred to the individual nodes themselves in the push-based approach. Here, messages are not distributed regularly, but only when necessary. In addition to this procedure, the pull-based approach disseminates a request for specific information into the network, after which other nodes send response messages. The last category represents cluster-based data distribution, where the network is divided into self-organized clusters using the concept of geocasting.

The number of use cases enabled by this technology is vast and continually growing. It encompasses safety-critical use cases such as "Cooperative Forward Collision Warning" and "Pre-Crash Sensing/Warning," as well as improvements to overall traffic flow through applications like "Green Light Optimal Speed Advisory." All of these, among others, are listed and explained in the "Car2X Manifesto" \cite{baldessari2007car}.

\section{Digital Twins}
\label{se:DigitalTwin}

"Digital Twin" is a newly introduced term in the field of IoT. Its goal is to simulate a system and mirror it into the digital world, and thus construct a Digital Twin. In this concept, changes to the real-world system will also apply as changes to the Digital Twin. However, unlike a Digital Shadow, changes to the twin will also apply to its real-world counterpart \cite{bazmohammadi2022microgrid}. This allows for enhanced integration of AI-based simulations and predictions. These AI algorithms can work on the Digital Twin and thus enhance the performance of the twinned system. It should also be noted that more traditional tools, such as simulation software, data analytics, and other optimization algorithms, can also be applied to the Digital Twin.

Piromalis et al. suggest that Digital Twins, in general, can be sorted into three categories \cite{piromalis2022digital}. The "Product Twin" is created as a digital prototype and used to identify possible problems beforehand using analytics tools and, in recent years, especially AI. In the automotive industry, this would commonly be a digital twin of a car model. A so-called "Process Twin" is used to replicate not an object but a certain process and optimize them by comparing different conditions. The Digital Twin of a whole production line would be a use case for this type. Combining the two earlier types is what makes a "System Twin". This, for example, could be the traffic management system for a whole city.

An example of a System Twin is provided by Wang et al. \cite{wang_2021}. Here, they designed a framework for a "Mobility Digital Twin", consisting of several Digital Twins including one of a human actor, vehicle, and a traffic system. Over the course of a literature review and based on their experiences, they also provide insight into the challenges such a system faces. Most of these are identical to the challenges Digital Twins face in general. One of the most pressing challenges is that of data transfer as well as management within cloud-based solutions. The amount of data required to keep a Digital Twin reliable is quite vast, and the implementation of fitting frameworks is challenging. Also, the cloud solution must be capable of handling real-time data streams. This plays into the general challenge that a real-time system requires, especially in safety-critical scenarios.

A general study about the troubles of implementing a holistic system is provided by Kherbache et al. \cite{gomez2021}. The paper especially iterates on the challenges of managing the complexity of a holistic network and keeping all the variables in check that might impact it. Also, especially in the automobile industry, security as well as privacy are big factors since Digital Twins are essentially monitoring everything in real-time and are sending the data. The general requirements for Digital Twins in regard to data confidentiality are formulated by Teisserenc et al. in "Project Data Categorization, Adoption Factors, and Non-Functional Requirements for Blockchain Based Digital Twins in the Construction Industry 4.0" \cite{teisserenc2021project}. Although they focus on a different industry as well as a more specific use case, most of the listed requirements still apply to Digital Twins in general and thus to those in the automobile industry. 
Furthermore, Wagner et al. \cite{wagner2019challenges} summarizes that another concrete challenge lies in the creation process of the Digital Twin itself. Since multiple domains need to converge in the engineering of the twin, there is a natural barrier to understanding as engineers have to grasp various scientific domains. They also suggest that the information flow needs to be efficiently designed and that in general an interface for the interdomain data exchange has to be formulated.

Looking ahead to the general future of Digital Twins, with the ongoing rapid advancements in the fields of AI and machine learning (ML), we can expect to see the capabilities of Digital Twins enhance even further. This progress will enable more accurate simulations and forecasts. Additionally, the emergence of 5G might enable more rapid and reliable data transfer, mitigating one of the major challenges. Moreover, integrating Digital Twins with blockchain technology might help lower the risks in the field of data security.
\





\section{IoT in Car Manufacturing}
Another significant part of the automobile industry is the manufacturing. In section \ref{se:DigitalTwin}, we already mentioned the possibility to create a Process Twin. In this section, we will further discuss technologies that enable the data collection and exchange that are required for such a twin to work, as well as a few other technologies applied to the manufacturing process \cite{lu2019potential}. The Radio Frequency Identification (RFID), which was first patented in 1983, has seen a rise in use in manufacturing in general. The technology itself uses readers that, in turn, use radio waves to read RFID-tags that store data and are placed on objects. Furthermore, the technology is also able to determine the position of a certain RFID tag by triangulating the position of the tag using multiple readers. This leads to many advantages provided by the technology, as shown in the case study undertaken by Kang et al., "Implementation of an RFID-Based Sequencing-Error-Proofing System for Automotive Manufacturing Logistics" \cite{kang2018implementation}. Some of the use cases that are also of relevance for the automobile industry include:

\begin{enumerate}
\item RFID for Just-in-Sequence Operations
\item RFID for Error Proofing
\item RFID for Real-Time Inventory Information
\end{enumerate}

RFID is not only used to keep track of the digital inventory of the manufacturer here but also to check if the right parts are applied to the manufacturing process in the right order, thus further expanding on the ever-present "just in time" policy of the automobile industry as well as mitigating the risk of applying a wrong part at the wrong time. As such, all of these use cases further the course of cost reduction. Nevertheless, the technology itself is not without certain challenges. Even though the costs of the technology have steadily decreased, implementation costs can be quite high, and the technology is prone to interferences and is limited in its reading range. Also, the problem of tag collisions as well as reader collisions is still persisting. This creates a need for an anti-collision protocol such as the one suggested by Liu et al. in his paper "A Novel Reader Anti-Collision Protocol Optimized by Minimal Reverse Weight for Large-Scale RFID Systems" \cite{liu2022novel}.

Furthermore, the concept of machine-to-machine communication (M2M) deserves mention. \cite{gundougan2021impact} Analogous to the car-to-car communication elaborated on in section \ref{Se:car2Car}, M2M describes the data exchange between machines. This technology significantly enhances the capabilities of the "just-in-sequence" operations common in car manufacturing by enabling real-time coordination and synchronization between different machines and systems. However, like car-to-car communication, M2M is not without its challenges. It can be prone to similar errors such as the "Broadcast Storm" problem, where the simultaneous broadcast of messages by multiple machines can lead to network congestion and data loss. As the field of M2M communication continues to evolve, ongoing research and development efforts are focused on addressing these issues to further optimize the efficiency and reliability of manufacturing processes.

A second rising topic related to the manufacturing process in general is that of "Predictive Maintenance"(PdM). PdM uses sensors and predictive data analytics to recognize possible equipment failure before it occurs and thus schedule maintenance accordingly. \cite{cinar2022predictive} In this context, the Product Twin mentioned in section \ref{se:DigitalTwin} as well as the concept of a Digital Shadow again are possibly relevant to allow an interface with AI. Further, et al. \cite{ersoz2022systematic} concludes at the end of his review of over 52 different studies that especially research in the field of decision trees, big data and cloud computing in combination with further IoT developments will drive the advancement in this field. The challenges associated with this technology still vary with the complexity of the monitored parts.


\section{Conclusion}
The automobile sector might be among the sectors most strongly influenced by the emergence as well as the development of the IoT. The technologies working under this term influence almost every part of the industry, be it production, operations, or day-to-day use of the product. The influence of IoT technologies in this industry has persisted for a long time with the developments in car-to-X communication and continues to bridge into newly emerging scientific fields such as AI via Digital Twins. But this newly arising digitalization, like with most developments in the field of computer science, comes with an ever-growing risk for our privacy and security. Not only do technical challenges lie on the road ahead, but also the ever-present troubles of data security and safety, with cars collecting and transferring more and more personal data.

Another general aspect that further automation always faces is that of social acceptance. Automation in the traffic sector as well as general automation in the manufacturing sector tends to be met with social disdain, since there is always the ethical component of the replacement of general workers. This aspect should not be ignored as it will have a significant impact on the general penetration of these technologies throughout the industry. Sufficient penetration will be necessary though to enable us with further field studies and drive research and innovation in general. It remains a sociopolitical problem that must be addressed in the future.

Regarding future research, it is important to reiterate the strong connection that the research field of IoT has to other research fields. It often acts as an enabler of other technologies and achieves this by connecting real-world systems to the cyber world. With the advancements of sensors, the quality of data can be enhanced. With the improvements of networks, the data throughput and reliability of IoT systems can be improved. Additionally, with the emergence of new analytic technologies, we can expect the analysis to be more precise. In the coming years, we can expect growth in all these fields, thereby increasing the relevance of IoT.

\bibliographystyle{IEEEtran}
\bibliography{bib} % Name of your BibTeX file without the .bib extension

\end{document}