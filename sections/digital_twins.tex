\label{se:DigitalTwin}

"Digital Twin" is a newly introduced term in the field of IoT. Its goal is to simulate a system and mirror it into the digital world, and thus construct a Digital Twin. In this concept, changes to the real-world system will also apply as changes to the Digital Twin. However, unlike a Digital Shadow, changes to the twin will also apply to its real-world counterpart \cite{bazmohammadi2022microgrid}. This allows for enhanced integration of AI-based simulations and predictions. These AI algorithms can work on the Digital Twin and thus enhance the performance of the twinned system. It should also be noted that more traditional tools, such as simulation software, data analytics, and other optimization algorithms, can also be applied to the Digital Twin.

Piromalis et al. suggest that Digital Twins, in general, can be sorted into three categories \cite{piromalis2022digital}. The "Product Twin" is created as a digital prototype and used to identify possible problems beforehand using analytics tools and, in recent years, especially AI. In the automotive industry, this would commonly be a digital twin of a car model. A so-called "Process Twin" is used to replicate not an object but a certain process and optimize them by comparing different conditions. The Digital Twin of a whole production line would be a use case for this type. Combining the two earlier types is what makes a "System Twin". This, for example, could be the traffic management system for a whole city.

An example of a System Twin is provided by Wang et al. \cite{wang_2021}. Here, they designed a framework for a "Mobility Digital Twin", consisting of several Digital Twins including one of a human actor, vehicle, and a traffic system. Over the course of a literature review and based on their experiences, they also provide insight into the challenges such a system faces. Most of these are identical to the challenges Digital Twins face in general. One of the most pressing challenges is that of data transfer as well as management within cloud-based solutions. The amount of data required to keep a Digital Twin reliable is quite vast, and the implementation of fitting frameworks is challenging. Also, the cloud solution must be capable of handling real-time data streams. This plays into the general challenge that a real-time system requires, especially in safety-critical scenarios.

A general study about the troubles of implementing a holistic system is provided by Kherbache et al. \cite{gomez2021}. The paper especially iterates on the challenges of managing the complexity of a holistic network and keeping all the variables in check that might impact it. Also, especially in the automobile industry, security as well as privacy are big factors since Digital Twins are essentially monitoring everything in real-time and are sending the data. The general requirements for Digital Twins in regard to data confidentiality are formulated by Teisserenc et al. in "Project Data Categorization, Adoption Factors, and Non-Functional Requirements for Blockchain Based Digital Twins in the Construction Industry 4.0" \cite{teisserenc2021project}. Although they focus on a different industry as well as a more specific use case, most of the listed requirements still apply to Digital Twins in general and thus to those in the automobile industry. 
Furthermore, Wagner et al. \cite{wagner2019challenges} summarizes that another concrete challenge lies in the creation process of the Digital Twin itself. Since multiple domains need to converge in the engineering of the twin, there is a natural barrier to understanding as engineers have to grasp various scientific domains. They also suggest that the information flow needs to be efficiently designed and that in general an interface for the interdomain data exchange has to be formulated.

Looking ahead to the general future of Digital Twins, with the ongoing rapid advancements in the fields of AI and machine learning (ML), we can expect to see the capabilities of Digital Twins enhance even further. This progress will enable more accurate simulations and forecasts. Additionally, the emergence of 5G might enable more rapid and reliable data transfer, mitigating one of the major challenges. Moreover, integrating Digital Twins with blockchain technology might help lower the risks in the field of data security.
\



