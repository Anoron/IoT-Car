Over the past 60 years since the first introduction of the Internet, several approaches were made to enhance connectivity between the internet and the real world. The ever-morphing terminology describing those attempts has changed over the years to finally arrive at "Internet of Things". Nowadays it encompasses a wide array of topics that touch on various other cornerstones of research in the field of computer sciences such as Automation, Artificial Intelligence (AI), and Networks, just to name a few. In the context of Industry 4.0, the topics collected under the umbrella term of IoT are expanded upon and are further developed depending on the specific industrial branch they are used in. In this paper, we will focus on the usage of IoT in the automobile industry. Here several developments in the fields of sensor technology and communication (such as 5G) that have been in the works for years have culminated, giving rise to a wide spectrum of new technologies. "Some of these advancements, such as automated driving resulting from enhanced Car2X communication, are highly visible to the public. Others, like the development and application of digital twins, have remained largely under the public's radar. The Digital Twins itself being a prime example of the interweaving of several topics under the terminology by interconnecting real-time data collection, transmission, analytics, and application through sensors, Network as well as AI. In the context of less publicly present achievements, we will also talk about how the IoT developments have influenced the manufacturing process itself.

To this end, as mentioned above, this paper will first focus on the communication of vehicles with other endpoints (C2X). This contains communication between vehicles (C2C) as well as the communication between a vehicle and the infrastructure (C2I). A lot of the use cases of this technology have been defined in the "Car-2-car communication consortium-manifesto" by Prof. Dr. Matthias Deegener et al \cite{baldessari2007car}. Even though the manifesto itself is from the year 2007, it still provides a valuable entry point into the understanding of the relevance and capabilities of the Car2X principle itself. Afterwards, we will discuss the potential and the current state of the research concerning Digital Twins and their uses in the construction as well in the use of vehicles using the paper "Digital Twins in the Automotive Industry: The Road toward Physical-Digital Convergence" by Dimitrios Piromalis \cite{piromalis2022digital} as a guiding source. Finaly, we will continue with a quick overview of the relevant applications in manufacturing before finishing with a discussion as well as a conclusion about the state of the research.