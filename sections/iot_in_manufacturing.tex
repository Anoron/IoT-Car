Another significant part of the automobile industry is the manufacturing. In section \ref{se:DigitalTwin}, we already mentioned the possibility to create a Process Twin. In this section, we will further discuss technologies that enable the data collection and exchange that are required for such a twin to work, as well as a few other technologies applied to the manufacturing process \cite{lu2019potential}. The Radio Frequency Identification (RFID), which was first patented in 1983, has seen a rise in use in manufacturing in general. The technology itself uses readers that, in turn, use radio waves to read RFID-tags that store data and are placed on objects. Furthermore, the technology is also able to determine the position of a certain RFID tag by triangulating the position of the tag using multiple readers. This leads to many advantages provided by the technology, as shown in the case study undertaken by Kang et al., "Implementation of an RFID-Based Sequencing-Error-Proofing System for Automotive Manufacturing Logistics" \cite{kang2018implementation}. Some of the use cases that are also of relevance for the automobile industry include:

\begin{enumerate}
\item RFID for Just-in-Sequence Operations
\item RFID for Error Proofing
\item RFID for Real-Time Inventory Information
\end{enumerate}

RFID is not only used to keep track of the digital inventory of the manufacturer here but also to check if the right parts are applied to the manufacturing process in the right order, thus further expanding on the ever-present "just in time" policy of the automobile industry as well as mitigating the risk of applying a wrong part at the wrong time. As such, all of these use cases further the course of cost reduction. Nevertheless, the technology itself is not without certain challenges. Even though the costs of the technology have steadily decreased, implementation costs can be quite high, and the technology is prone to interferences and is limited in its reading range. Also, the problem of tag collisions as well as reader collisions is still persisting. This creates a need for an anti-collision protocol such as the one suggested by Liu et al. in his paper "A Novel Reader Anti-Collision Protocol Optimized by Minimal Reverse Weight for Large-Scale RFID Systems" \cite{liu2022novel}.

Furthermore, the concept of machine-to-machine communication (M2M) deserves mention. \cite{gundougan2021impact} Analogous to the car-to-car communication elaborated on in section \ref{Se:car2Car}, M2M describes the data exchange between machines. This technology significantly enhances the capabilities of the "just-in-sequence" operations common in car manufacturing by enabling real-time coordination and synchronization between different machines and systems. However, like car-to-car communication, M2M is not without its challenges. It can be prone to similar errors such as the "Broadcast Storm" problem, where the simultaneous broadcast of messages by multiple machines can lead to network congestion and data loss. As the field of M2M communication continues to evolve, ongoing research and development efforts are focused on addressing these issues to further optimize the efficiency and reliability of manufacturing processes.

A second rising topic related to the manufacturing process in general is that of "Predictive Maintenance"(PdM). PdM uses sensors and predictive data analytics to recognize possible equipment failure before it occurs and thus schedule maintenance accordingly. \cite{cinar2022predictive} In this context, the Product Twin mentioned in section \ref{se:DigitalTwin} as well as the concept of a Digital Shadow again are possibly relevant to allow an interface with AI. Further, et al. \cite{ersoz2022systematic} concludes at the end of his review of over 52 different studies that especially research in the field of decision trees, big data and cloud computing in combination with further IoT developments will drive the advancement in this field. The challenges associated with this technology still vary with the complexity of the monitored parts.
